\chapter{Current state of Quantum Computing}
There is probably no single point in space-time where the concept of \textit{quantum computing} emmerged for the very first time. In the 70' Paul Benioff was investigating the topic and in year 1980 published a foundational article \textit{The computer as a physical system: A microscopic quantum mechanical Hamiltonian model of computers as represented by Turing machines}[ref]. But it took celebrity-physisist and Nobel Prize winner Richard Feynmann to mark the beginning of the discipline (for sure anegdoticaly). He discussed the topic in 1981 talk \textit{Simulating physics with computers} during a small conference on the topic of \textit{Physics of Computation}. Present were also Paul Benioff and other scientists whose work contributed to the birth of the discipline: Rolf Landauer and Tom Toffoli [https://mitendicotthouse.org/physics-computation-conference/]. In his foundational talk, Feynmann indicated that for an efficient simullation of quantum phenomena, a different kind of computer was needed. The reasoning behind the idea is that quantum systems description grows expotentialy with the number of particles involved. A system large enough will quickly overwhelm any classical computer. The conclusion of the talk:
\begin{quote}
Nature isn’t classical, dammit, and if you want to make a simulation of Nature, you’d better make it quantum mechanical, and by golly it’s a wonderful problem because it doesn’t look so easy.
\end{quote}
became famous and is often treated as a spark that ignited the discipline.
[https://twitter.com/preskill/status/838819799857684481?lang=en]
[https://www.nature.com/articles/nphys2258]
[https://www.economist.com/leaders/2019/09/28/google-claims-to-have-demonstrated-quantum-supremacy][With those words, in 1981, Richard Feynman, an American physicist, introduced the idea that, by harnessing quantum mechanics, it might be possible to build a new kind of computer, capable of tackling problems that would cause a run-of-the-mill machine to choke.]
[https://thequantuminsider.com/2021/11/25/global-chip-crisis-quantum-computing-could-be-the-answer/]


Quantum computing has seen a rapid growth in recent years. Major players in the tech industry, governments and universities invest their time, money and effort.
\section{Gate model}
Some text about the gate model computers

\subsection{Error correction}
Some for subsection of error correction
- Google quantum supremacy?
- biggest chip so far

\section{Problems of practical QC - NISQ}
Some text about NISQ
\section{Quantum Annealing}
Quantum annealing has some interesting properties. It is analog in nature and because the computation happens in a ground state it is unaffected by decoherence \citep{McGeoch}. 
\cite{McGeoch}