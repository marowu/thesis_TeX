\subsection{Lenz-Ising Model}
The model is most often called \textit{Ising model} after Ernst Ising. In his brief paper from 1925 he clearly states that he developed a concept first proposed by his research director --- Wilhelm Lenz in 1920\cite{ising_beitrag_1925}. Hence I follow naming used by Brush and Niss\cite{brush_history_1967}\cite{niss_history_2005}. Model proposed by Lenz assumed a magnetic material to be constituted of elementary micromagnets (as proposed in 1907 by P.~Weiss) but with an additional constraint that each micromagnet is only able to take two positions "up" and "down". Ising developed the model by considering nearest-neighbour interactions in a linear chain of micromagnets. Exact solution obtained by Ising showed that one-dimensional model was not capable of explaining the ferromagnetism. Erroneously, Ising concluded that his result was valid in three dimensions:
\begin{quote}
Es ist gezeigt, das für lineare Ketten und gewisse Modifikationen kein Ferromagnetismus unter den obigen Voraussetzungen entsteht. Es besteht die Vermutung, daß auch beim dreidimensionalen analogen Modell ein ferromagnetisches Verhalten nicht erzielt werden. \textit{(It is shown that for linear chains and certain modifications no ferromagnetic effect was achieved under the above conditions. It is assumed that also in case of a three-dimensional model, ferromagnetic behaviour will not be achieved)}\cite{ising_beitrag_1925}
\end{quote}
The ideas proposed by Lenz and Ising survived despite the lack of apparent success. In 1936 Rudolf Peierls showed that for a two-dimensional case the model indeed is capable of explaining the ferromagnetism. Progress in other areas of study such as binary alloys and transition points resulted in formulation of descriptions mathematically equivalent to Lenz-Ising model. To describe any of these phenomena in a system, one has to consider joint action of units the system is composed of, hence the term \textit{cooperative phenomena}. Finally in 1944 Rals Onsager published a paper titled \textit{A Two-Dimensional Model with an Order-Disorder Transition}. He provided a mathematical proof that two-dimensional Lenz-Ising model is capable of explaining ferromagnetism. The history of Lenz-Ising model is very interesting and often surprising. Please refer to Brush \cite{brush_history_1967} and Niss\cite{niss_history_2005} for a detailed description.

The model, however basic and idealised, proved to offer a great trade-off between realism and mathematical usability. This resulted in its great success both in terms of citations (thousands for papers of Ising and Onsager) but also a growing number of areas it is used in. One of proofs of this success is the important role the model currently plays in the area of combinatorial optimisation.

\hl{the model below is a hamiltonian. How did the original concept end up being hamiltonian?}

\hl{modern version of the model description}

\[H(\sigma) = -\sum_{\langle i~j\rangle} J_{ij} \sigma_i \sigma_j - \mu \sum_j h_j \sigma_j,\]

Many combinatorial optimization problems have well documented Lenz-Ising formulations. Please refer to \cite{lucas_ising_2014} for a comprehensive collection of such formulations for many NP problems.