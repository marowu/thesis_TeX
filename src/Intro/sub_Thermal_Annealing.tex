\subsection{Thermal Annealing} \label{sub:thermal_annealing}
Traditionally annealing refers to thermal annealing, a metallurgical process known to humanity for centuries. Annealing reduces hardness and increases ductility of a (most often) metal peace, making it more workable. The peace is heated until glowing (for steel) and kept in high temperature. The temperature increases the rate at which atoms diffuse in the metal, eradicating dislocations of the crystalline structure, hence moving towards equilibrium state - lower internal stresses. Lower number of dislocations results in higher ductility. The process of stress-relief is spontaneous, so it also progresses at room temperature, although at a very slow pace. It is the heat that increases the energy of thermal fluctuations, making transitions more probable. This concept is the basis on which, by analogy, optimisational algorithms were build. Annealing may also be applied to glass, with similar purposes and process. As glass has no crystalline structure, the mechanics differ \cite{wikipedia_contributors_annealing_2021}.

A process with opposite effects is quenching, where hot material is cooled rapidly. This results in sub-optimal configuration of the material. Remaining stresses make the material harder but more brittle \cite{wikipedia_contributors_quenching_2021}.